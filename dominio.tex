\section{Studio del dominio}

%TODO Riempire di citazioni

\subsection{Il fallimento}
Il fallimento è lo strumento attraverso il quale si supera l'inattività o la volontà contraria al soddisfacimento delle obbligazioni assunte dal debitore, rispetto alle quali il suo intero patrimonio svolge una funzione di garanzia.

Si svolge nell'interesse dei creditori e nel rispetto della par condicio creditorum innanzi al Tribunale del luogo ove ha sede principale l'imprenditore.

Lo stato d'insolvenza si manifesta con inadempimenti o altri fatti esteriori, i quali dimostrino che il debitore non è più in grado di soddisfare regolarmente le proprie obbligazioni.

\subsection{Soggetti interessati}

\subsubsection{Tribunale fallimentare}
L'organo principale investito dell'intera procedura fallimentare. Nomina, revoca e sostituisce gli altri organi della procedura, quando non è prevista la competenza del giudice delegato.

\subsubsection{Imprenditore}
Sono soggetti alle disposizioni sul fallimento gli imprenditori che esercitano un'attività commerciale.

\subsubsection{Giudice delegato}
Il giudice delegato esercita funzioni di vigilanza e di controllo sulla regolarità della procedura.

\subsubsection{Curatore fallimentare}
Il curatore fallimentare è colui che ha l’amministrazione del patrimonio fallimentare e compie tutte le operazioni della procedura fallimentare sotto la vigilanza del giudice delegato e del comitato dei creditori, nell’ambito delle funzioni ad esso attribuite.


\subsubsection{Comitato dei creditori}
Il comitato dei creditori vigila sull'operato del curatore e ne propone la revoca, autorizza gli atti, esprime pareri.
I membri possono svolgere ispezioni sulle scritture contabili e sui documenti della procedura. È nominato dal giudice delegato, sentiti il curatore e i creditori stessi.


\subsection{Istanza di fallimento}
L'istanza di fallimento può essere presentata dall'imprenditore stesso, da un pubblico ministero oppure da uno o più creditori.

\subsection{Iscrizione al passivo}

\subsubsection{Tipologia di iscrizione}


\paragraph{COME SI FA}




\subsection{Studio del dominio e fonti utilizzate}


\subsection{Individuazione delle componenti di un sistema}
\subsection{Individuazione dei bisogni dell’utente}