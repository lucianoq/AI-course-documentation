%\documentclass[a4paper,12pt,oneside,onecolumn]{article}
\usepackage[utf8]{inputenc}
\usepackage[italian]{babel}
\usepackage{hyperref}
\usepackage[round,numbers]{natbib}
\usepackage{listings}
%\usepackage{graphicx}
%\usepackage{amsmath}
%\usepackage[T1]{fontenc}
%\usepackage[scaled=0.9]{beramono}
%\usepackage{microtype}
\usepackage{color}
\usepackage{xcolor}
%\usepackage{float}
\usepackage{lipsum}
\usepackage[noadvisor,swapnames]{frontespizio}

\definecolor{light-gray}{gray}{0.95}

\lstset{ %
  backgroundcolor=\color{light-gray},
  basicstyle=\footnotesize,
  breakatwhitespace=false,         % sets if automatic breaks should only happen at whitespace
  breaklines=true,                 % sets automatic line breaking
  captionpos=b,                    % sets the caption-position to bottom
  commentstyle=\color{green},    % comment style
  deletekeywords={...},            % if you want to delete keywords from the given language
  escapeinside={\%*}{*)},          % if you want to add LaTeX within your code
  extendedchars=true,              % lets you use non-ASCII characters; for 8-bits encodings only, does not work with UTF-8
  frame=single,                    % adds a frame around the code
  keepspaces=true,                 % keeps spaces in text, useful for keeping indentation of code (possibly needs columns=flexible)
  keywordstyle=\color{blue},       % keyword style
  language=Prolog,
  morekeywords={*,...},            % if you want to add more keywords to the set
  numbers=left,
  numbersep=10pt,                   % how far the line-numbers are from the code
  numberstyle=\scriptsize1\color{black}, % the style that is used for the line-numbers
  rulecolor=\color{black},
  showspaces=false,                % show spaces everywhere adding particular underscores; it overrides 'showstringspaces'
  showstringspaces=false,          % underline spaces within strings only
  showtabs=false,                  % show tabs within strings adding particular underscores
  stepnumber=1,
  stringstyle=\color{magenta},     % string literal style
  tabsize=3,
  title=\lstname
}




\begin{document}

    \begin{frontespizio}
        \Universita {Bari ``Aldo Moro"}
        \Logo [2cm]{./images/logo}
        \Dipartimento{Informatica}
        \Corso [Laurea]{Informatica Magistrale}
        \Titoletto {Documentazione di progetto di Intelligenza Artificiale}
        \Titolo{Text Extraction d in ambito giuridico}
        \NCandidati{Studenti}
        \Candidato{Luciano Quercia}
        \Candidato{Simone Rutigliano}
        \Annoaccademico {2013-2014}
        \Margini{4cm}{4.5cm}{4cm}{4cm}
    \end{frontespizio}

\section{Introduzione}

\subsection{Text Analytics}
\nocite{wiki:textMining}
\nocite{gartner:textAnalytics}
\nocite{expertsystem:textAnalytics}
La Text Analytics è un processo per identificare ed estrarre entità (persone, luoghi, organizzazioni, indirizzi, valute, etc.) da dati non strutturati (documenti, pagine web, email, PDF) attraverso l'utilizzo ibrido di tecniche linguistiche,statistiche e di machine learning. Questo processo generalmente viene utilizzato per diversi scopi:
\begin{itemize}
	\item Summarization: Si riassume il documento in questione individuandone le parole chiave;
	\item Sentiment Analysis: si identificano, estraggono, etichettano e rielaborano le informazioni legate ad uno o più brand con l'obiettivo di determinare sia l'attitudine di chi ha pubblicato un contenuto legato alla marca stessa sia la polarità contestuale del contenuto (positiva, neutra, negativa);
	\item Esplicativa: Capire a cosa si vuole arrivare con un particolare documento;
	\item Investigativa: Capire la causa di uno specifico problema;
	\item Classificazione: Classificare il testo in un particolare argomento.
\end{itemize}

L'utilizzo di un sistema esperto per questa tipologia di processi può risultare molto vantaggiosa in quanto diventa possibile automatizzare l'estrazione delle informazioni dal testo non solo in base alla lingua con cui è stato realizzato il testo, ma anche grazie alla base di conoscenza del dominio.


%http://en.wikipedia.org/wiki/Text_mining#Text_mining_and_text_analytics

%The term text analytics describes a set of linguistic, statistical, and machine learning techniques that model and structure the information content of textual sources for business intelligence, exploratory data analysis, research, or investigation.[1] The term is roughly synonymous with text mining; indeed, Ronen Feldman modified a 2000 description of "text mining"[2] in 2004 to describe "text analytics."[3] The latter term is now used more frequently in business settings while "text mining" is used in some of the earliest application areas, dating to the 1980s,[4] notably life-sciences research and government intelligence.

%The term text analytics also describes that application of text analytics to respond to business problems, whether independently or in conjunction with query and analysis of fielded, numerical data. It is a truism that 80 percent of business-relevant information originates in unstructured form, primarily text.[5] These techniques and processes discover and present knowledge – facts, business rules, and relationships – that is otherwise locked in textual form, impenetrable to automated processing.

%http://www.gartner.com/it-glossary/text-analytics
%Text analytics is the process of deriving information from text sources. It is used for several purposes, such as: summarization (trying to find the key content across a larger body of information or a single document), sentiment analysis (what is the nature of commentary on an issue), explicative (what is driving that commentary), investigative (what are the particular cases of a specific issue) and classification (what subject or what key content pieces does the text talk about).


%www.expertsystem.net/solutions/text-analytics

%Text Analytics
%Extract value from unstructured information
%
%Text analytics is the process of identifying and extracting entities (people, places, organizations, measures, addresses) and other concepts (for example vehicles, buildings, chemical substances, weapons, etc.) within unstructured data (documents, web pages, emails, PDFs), fueling an organization’s analytical processes through the discovery of strategic links and patterns between information or data points.

%Why Semantic Technology?
%Through the automatic understanding of the meaning of words and the identification of relationships between concepts in a text, Expert System’s semantic technology brings an unprecedented level of precision and recall to the extraction of information and intelligence from the unstructured part of Big Data.
%
%Traditional text analytics applications are limited to analyzing text by forcing users to continuously manage lists. Expert System’s Cogito semantic technology leverages deep semantic analysis and the vast knowledge embedded in its semantic network to extract a virtually unlimited number of elements quickly and easily. Cogito features:
%
%Effective management of big data and growing volumes of unstructured information.
%Automatic identification and extraction of more than 20 standard entities.
%Customized selection and extraction of entities.
%Advanced exploration and analysis of concepts and relationships between documents.

\subsection{Scopo del progetto}
L'obiettivo del progetto sarà quello di creare un sistema esperto che sia in grado di estrarre le informazioni salienti da documenti giuridici inerenti istanze fallimentari.
\section{Studio del dominio}
\subsection{Descrizione del dominio}
\subsection{Studio del dominio e fonti utilizzate}
\subsection{Individuazione delle componenti di un sistema}
\subsection{Individuazione dei bisogni dell’utente}
\section{Progettazione}
\subsection{Concettualizzazione}
\subsection{Oggetti del dominio}
\subsection{Individuazione delle funzioni}
\subsection{Individuazione delle relazioni}
\subsection{Dalla concettualizzazione alle regole}
\subsection{Strategia risolutiva}
\section{Implementazione}

\subsection{Architettura del sistema}
Per preservare l'indipendenza delle componenti del sistema, abbiamo realizzato dei moduli Prolog che espongano pubblicamente solo i predicati necessari.

Il file \verb+main.pl+ costituisce l'entry-point del programma, l'unico file necessario da importare nell'interprete per l'esecuzione.

Come anticipato nella sezione~\ref{sec:Progettazione}, il sistema è composto principalmente da un \emph{lexer} e da un \emph{tagger}.

Lo schema delle dipendenze è mostrato in figura~\ref{fig:moduli}.

\begin{figure}[h!tbp]
\includegraphics[width=\textwidth]{img/struttura_moduli.png}
\label{fig:moduli}
\end{figure}

\subsection{Lexer}
Il compito del \emph{lexer} sarà quello di normalizzare la stringa contenente il documento da analizzare in una lista di token su cui poi il tagger dovrà fare le sue analisi. Per fare ciò, bisognerà nell'ordine: 
\begin{itemize}
    \item ripulire la stringa di stopchars;
    \item separare i caratteri speciali (punto, virgola, euro e chiocciola) da eventuali caratteri a cui sono associati;
    \item eliminare eventuali spazi bianchi superflui causati da eliminazioni di caratteri fatte in precedenza;
    \item rendere case insensitive la stringa ripulita.
\end{itemize}
Di seguito il predicato lexer:

\begin{prologcode}
lexer(String,ListToken) :-
    filter_stopchars(String, A),
    clean_chiocciola(A,B),
    clean_punto(B,C),
    clean_virgola(C,D),
    clean_euro(D,E),
    strip_spaces(E, F),
    atom_codes(G, F),
    atomic_list_concat(H,' ', G),
    maplist(downcase_atom, H, ListToken).
\end{prologcode}

\subsection{Tagger}
Il compito del \emph{tagger} invece, sarà quello di individuare in maniera deterministica le componenti salienti del testo cosi come detto in precedenza nel paragrafo \ref{tagger}.

Di seguito verrà riportato il predicato tagger:

\begin{prologcode}
tagger(ListToken,ListTagged) :-
    tag_persona(ListToken,A),
    tag_indirizzo(A,A1),
    tag_mail(A1,B),
    tag_money(B,C),
    tag_date(C,D),
    tag_comune(D,E),
    tag_cf(E,F),
    tag_numero_telefono(F,G),
    filter_stopwords(G,ListTagged).
\end{prologcode}

\subsection{Persone}
Al fine di individuare tutte le persone fisiche, con i rispettivi ruoli, all'interno del documento, è stato necessario utilizzare la tecnica del divide et impera. In questo modo, in prima battuta, sono stati individuati contemporaneamente tutte le possibili persone e tutti i probabili ruoli che una persona può avere, quali ad esempio (colui che creditore verso l'entità fallita, commissario, giudice, curatore, avvocato,etc.).

Per verificare se una sequenza di token può considerarsi essere una persona, sono stati presi in esame tutte le possibili combinazioni di nomi e cognomi che una persona italiana può avere secondo lo stato italiano; si potranno avere infatti, fino ad un massimo di due cognomi concatenati a tre nomi o la sua commutazione. Inoltre, affinché fosse verificato che un token sia un nome o un cognome, è stata utilizzata una base di conoscenza prolog contenente un buon quantitativo di nomi e cognomi italiani.

Successivamente si andranno a unificare i ruoli trovati con le persone più vicine a quei ruoli, creando cosi l'entità finale composta da persona e ruolo.
Inoltre, il predicato terrà anche in considerazione il fatto che in un documento ci sia la probabilità che la persona, oltre al ruolo abbia anche uno o più aggettivi a lui associato (e.g. Ill.mo Giudice Mario Bianchi); in questo caso, gli aggettivi non verranno presi in considerazione nel tag finale.

Inoltre, è stato realizzato il predicato \emph{tag\_indirizzi} che avrà il compito di evitare che vengano taggate persone presenti in indirizzi postali.

Di seguito il predicato \emph{tag\_persona} che permette la realizzazione del tag descritto:

\begin{prologcode}
tag_persona(List,ListTagged) :-
    tag_aggettivo(List,A),
    tag_ruolo(A,B),
    strip_aggettivi(B,C),
    tag_nome(C,D),
    tag_titolo(D,ListTagged).
\end{prologcode}

\subsection{Richiesta di denaro}
La seconda entità principale che il tagger andrà a ricercare all'interno del documento sarà il quantitativo di denaro che il creditore andrà a richiedere al debitore. Per fare ciò, si seguirà la stessa linea di pensiero usata per il tag precedente, nella fattispecie, in una prima fase si andranno ad individuare sia tutte le possibili cifre di denaro presenti nel documento, sia tutte le tipologie di richieste che si potranno fare (chirografario, privilegiato e totale); nella seconda fase, invece, si andranno ad unificare le due entità più vicine di questi due tipi.

Per quanto riguarda l'individuazione delle cifre di denaro, sono state prese in considerazione tutte le possibili combinazioni di valore e valuta (e.g. 20\texteuro, \textdollar 20, 20 eur, etc.).

Invece, per l'individuazione delle tipologie di richieste, si è preferito uniformare le tipologie ad una singola parola per tipologia (e.g. chirografario, chirografaria, chirografo, chirografa, chiro e chir verranno tutte ricondotte al termine \emph{chirografario}).

Di seguito il predicato \emph{tag\_money} in grado di eseguire le operazioni descritte nel paragrafo.

\begin{prologcode}
tag_money(A,F) :- 
    tag_currency(A,B),
    tag_chirografario(B,C),
    tag_privilegiato(C,D),
    tag_totale(D,E),
    tag_richiesta(E,F).
\end{prologcode}

\subsection{Informazioni aggiuntive}

%TODO
%tag\_mail
%tag\_date
%tag\_comune
%tag\_cf
%tag\_numero\_telefono
%filter\_stopwords

%TODO JPL DEPRECATED LIST

\subsection{Le regole}
\subsection{L'inferenza}
\section{Componenti aggiuntive}
\subsection{Gestione dell'incertezza}
\subsection{Come e perché}
\subsection{Interazione con l'utente}
\subsection{Alcune funzioni utili}
\section{Descrizione del sistema}

Il sistema presenta un'interfaccia grafica in grado di permettere l'interazione con il core del sistema scritto in Prolog, dando cosi la possibilità a qualunque tipo di stakeholders del sistema di utilizzarlo senza la necessità di dover interagire con il terminale, rendendo cosi le informazioni più leggibili e usabili; Oltre a questo, la creazione dell'interfaccia permette anche di evitare possibili errori dattilografici che si potrebbero avere in caso di interazione con il terminale.

Gli elementi principali dell’interfaccia, con i relativi pulsanti, la cui funzione e uso verranno descritti in dettaglio nel prossimo paragrafo, sono i seguenti (nella figura _, i numeri delle aree corrispondono alle rispettive funzioni assegnate enumerate
nell’elenco seguente):
1. Sezione di inserimento del documento da cui si devono estrarre le informazioni
2. Sezione di scelta delle informazioni da estrarre
2. Sezione di visualizzazione dei risultati ottenuti

\subsection{Sezioni ed operazioni disponibili}
    \subsubsection{Inserimento}
    In questa sezione viene data la possibilità di inserire il documento testuale da cui si vogliono estrarre le informazioni; con questa operazione non si fa altro che asserire un documento da dover poi essere processato dal core prolog del sistema.
    \subsubsection{Scelta dei tag}
    In questa sezione si da la possibilità all'utente di filtrare i tag da voler estrarre dal documento attraverso la selezione/deselezione della checkbox corrispondente al tag; con questa operazione si vanno a selezionare quali saranno i tag che il core prolog deve etichettare nel documento.
    \subsubsection{Visualizzazione}
    In questa schermata invece verranno mostrate le informazioni che l'utente ha deciso di estrarre dal documento; inoltre le diverse tipologie di tag saranno evidenziate diversamente l'uno dall'altro attraverso l'ausilio di una colorazione dei tag. 

    \subsubsection{Reset delle condizioni iniziali}
    Tramite il pulsante “Reset” si ripristinano le condizioni iniziali del sistema.
    Viene ripristinato lo stato iniziale sia dell’interfaccia che del modulo Prolog, la cui working memory viene liberata dei fatti, quindi dai    documenti, appresi durante l’esecuzione.

\subsection{Interazione con l’utente}
L’interazione avviene principalemente tramite l’interfaccia grafica descritta nelle sezioni seguenti. La comunicazione fra il lato Java e quello Prolog sono riportate nella finestra di terminale da cui si `e lanciato l’eseguibile. Inoltre, viene data la possibilità di utilizzare il sistema anche direttamente dalla linea di comando dell’interprete Prolog, caricando il file main.pl. Le funzionalità offerte rimangono invariate. 

\subsection{Esempio di interazione}

\subsection{Utilizzo della funzione spiega domanda}

\subsection{Presentazione dei risultati}

\subsection{Utilizzo della funzione spiega ragionamento}


\bibliographystyle{abbrvnat}
\bibliography{mybib}

\end{document}