\section{Sperimentazione}
\label{experiment}

La sperimentazione è stata condotta utilizzando il dataset MovieLens, già noto in letteratura.
Il dataset è composto da 518 film e 613 utenti. Ciascun utente ha votato più film, con una votazione che va da 1 a 5.

Il protocollo sperimentale utilizzato nel progetto, prevede i seguenti passi:
\begin{enumerate}
\item Split del set di ogni utente secondo la proporzione 70/30 (il 70\% del set verrà utilizzato come Training Set, il restante 30\% come Test Set);
\item Per ogni utente:
    \begin{enumerate}
        \item Si creano i due profili utente (pesato e non pesato) utilizzando solo i dati presenti nel suo Training set;
        \item Si creano le liste di raccomandazioni, contenente la totalità dei film, ordinate per importanza, secondo le diverse tipologie di distanza definite nel capitolo \ref{measures};
    \end{enumerate}
\item $\forall k \in (5,10,20,50,100)$ , calcolo della Precisione $P_k$ (\ref{precision}), e del valore MRR (\ref{MRR}) tenendo conto anche delle diverse distanze calcolate e dei diversi profili costruiti in precedenza;
\end{enumerate}
Formalmente, possiamo dire che, l'esperimento controllato che verrà eseguito sarà un esperimento avente $n$ fattori con $m$ livelli, in particolare avremo 3 fattori:
\begin{itemize}
\item \textbf{Distanza} con 4 livelli: 
    \begin{itemize}
        \item \emph{Passant D} (\ref{PassantD}),
        \item \emph{Passant C} (\ref{PassantC}),
        \item \emph{Passant DW} (\ref{PassantDW}),
        \item \emph{Passant CW} (\ref{PassantCW});
    \end{itemize}
\item \textbf{Profilo} con 2 livelli (\ref{profili}) : 
    \begin{itemize}
        \item \emph{Pesato},
        \item \emph{Non Pesato};    
    \end{itemize}
\item \textbf{Numero di raccomandazioni $k$} con 5 livelli:
    \begin{itemize}
        \item \emph{5},
        \item \emph{10},
        \item \emph{20},
        \item \emph{50},
        \item \emph{100}.
    \end{itemize}
\end{itemize}
\subsection{Metriche}
\label{metriche}
Al fine di valutare la bontà del sistema creato abbiamo calcolato due tipologie di metriche:
\begin{itemize}
\item Precision
\item MRR
\end{itemize}
Per entrambe le tipologie sono stati presi in esame due punti di vista differenti (\emph{macroscopico} e \emph{microscopico}) ed inoltre, per ogni punto di vista, si è lavorato su due versioni differenti dello stesso dataset (\emph{epurato} e \emph{non epurato}). Questo è stato necessario al fine di migliorare le metriche relative alle raccomandazioni in quanto, nella valutazione, si vanno a considerare solo ed esclusivamente film presenti nel test set (\emph{epurato}) e non nell'intera totalità dei film presenti, di cui nella maggior parte dei casi (film non presenti nel test set) non conosciamo l'opinione dell'utente su quel particolare film (\emph{non epurato}).

\subsubsection{Precision}
\label{precision}
In un processo di classificazione statistica, la \emph{precisione} per una classe è il numero di \emph{true positive} (il numero di oggetti etichettati correttamente come appartenenti alla classe) diviso il numero totale di elementi etichettati come appartenenti alla classe (la somma di \emph{true positive} e \emph{false positive}, che sono oggetti etichettati erroneamente come appartenenti alla classe).
Inoltre i termini true positive, true negative, false positive e false negative sono usati per confrontare la classificazione di un oggetto (l’etichetta di classe assegnata all’oggetto da un classificatore) con la corretta classificazione desiderata (la classe a cui in realtà appartiene l’oggetto).
\begin{equation*}
Precision =\frac{TP}{TP+FP}
\end{equation*}

\paragraph{MicroPrecision}
Nel calcolo della microprecisione, è necessario prendere in considerazione i singoli valori di true positive e di false positive di ogni singolo utente come descritto nella formula:
\begin{equation*}
MicroPrecision =\frac{\sum\limits_{u\in U}^{}TP_u}{\sum\limits_{u\in U}^{}TP_u+\sum\limits_{u\in U}^{}FP_u}
\end{equation*}

\paragraph{MacroPrecision}
Per quanto riguarda invece il calcolo della macroprecisione, si va a prendere in considerazione il valor medio delle singole precisioni definite a livello di utente. Formalmente può essere descritto secondo la formula:
\begin{equation*}
MacroPrecision =\frac{1}{|U|}\sum\limits_{u\in U}{Precision_u}
\end{equation*}

\subsubsection{MRR}
\label{MRR}
Il Mean Reciprocal Rank (MRR) è un indice statistico per valutare un processo che produce una lista di possibili risposte ad una interrogazione (query), ordinate per probabilità di correttezza.
\begin{equation*}
MRR = \frac{1}{|Q|}\sum_{i=1}^{Q}{\frac{1}{Rank_i}}
\end{equation*}

\paragraph{MicroMRR}
Nel calcolo della microMRR, è necessario prendere in considerazione i singoli ranking delle raccomandazioni fatte ad ogni utente come descritto nella formula:
\begin{equation*}
MicroMRR =\frac{\sum\limits_{i=1}^{|R|}{\frac{relevant(r_i)}{i}}}{\sum\limits_{i=1}^{|R|}{\frac{1}{i}}} \qquad relevant(r_i)=\begin{cases} 1 & \mbox{se }r_i\mbox{ è rilevante} \\ 0 & \mbox{altrimenti}
\end{cases}
\end{equation*}
\paragraph{MacroMRR}
Per quanto riguarda invece il calcolo della macroMRR, si va a prendere in considerazione il valor medio dei singoli MRR definiti a livello di utente. Formalmente può essere descritto secondo la formula:
\begin{equation*}
MacroMRR =\frac{1}{|U|}\sum_{u\in U}{MRR_u}
\end{equation*}

