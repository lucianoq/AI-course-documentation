\section{Introduzione}

\subsection{Text Analytics}
\nocite{wiki:textMining}
\nocite{gartner:textAnalytics}
\nocite{expertsystem:textAnalytics}
La Text Analytics è un processo per identificare ed estrarre entità (persone, luoghi, organizzazioni, indirizzi, valute, etc.) da dati non strutturati (documenti, pagine web, email, PDF) attraverso l'utilizzo ibrido di tecniche linguistiche,statistiche e di machine learning. Questo processo generalmente viene utilizzato per diversi scopi:
\begin{itemize}
	\item \emph{Summarization}: Si riassume il documento in questione individuandone le parole chiave;
	\item \emph{Sentiment Analysis}: si identificano, estraggono, etichettano e rielaborano le informazioni legate ad uno o più brand con l'obiettivo di determinare sia l'attitudine di chi ha pubblicato un contenuto legato alla marca stessa sia la polarità contestuale del contenuto (positiva, neutra, negativa);
	\item \emph{Esplicativa}: Capire a cosa si vuole arrivare con un particolare documento;
	\item \emph{Investigativa}: Capire la causa di uno specifico problema;
	\item \emph{Classificazione}: Classificare il testo in un particolare argomento.
\end{itemize}

L'utilizzo di un sistema esperto per questa tipologia di processi può risultare molto vantaggiosa in quanto diventa possibile automatizzare l'estrazione delle informazioni dal testo non solo in base alla lingua con cui è stato realizzato il testo, ma anche grazie alla base di conoscenza del dominio.


%http://en.wikipedia.org/wiki/Text_mining#Text_mining_and_text_analytics

%The term text analytics describes a set of linguistic, statistical, and machine learning techniques that model and structure the information content of textual sources for business intelligence, exploratory data analysis, research, or investigation.[1] The term is roughly synonymous with text mining; indeed, Ronen Feldman modified a 2000 description of "text mining"[2] in 2004 to describe "text analytics."[3] The latter term is now used more frequently in business settings while "text mining" is used in some of the earliest application areas, dating to the 1980s,[4] notably life-sciences research and government intelligence.

%The term text analytics also describes that application of text analytics to respond to business problems, whether independently or in conjunction with query and analysis of fielded, numerical data. It is a truism that 80 percent of business-relevant information originates in unstructured form, primarily text.[5] These techniques and processes discover and present knowledge – facts, business rules, and relationships – that is otherwise locked in textual form, impenetrable to automated processing.

%http://www.gartner.com/it-glossary/text-analytics
%Text analytics is the process of deriving information from text sources. It is used for several purposes, such as: summarization (trying to find the key content across a larger body of information or a single document), sentiment analysis (what is the nature of commentary on an issue), explicative (what is driving that commentary), investigative (what are the particular cases of a specific issue) and classification (what subject or what key content pieces does the text talk about).


%www.expertsystem.net/solutions/text-analytics

%Text Analytics
%Extract value from unstructured information
%
%Text analytics is the process of identifying and extracting entities (people, places, organizations, measures, addresses) and other concepts (for example vehicles, buildings, chemical substances, weapons, etc.) within unstructured data (documents, web pages, emails, PDFs), fueling an organization’s analytical processes through the discovery of strategic links and patterns between information or data points.

%Why Semantic Technology?
%Through the automatic understanding of the meaning of words and the identification of relationships between concepts in a text, Expert System’s semantic technology brings an unprecedented level of precision and recall to the extraction of information and intelligence from the unstructured part of Big Data.
%
%Traditional text analytics applications are limited to analyzing text by forcing users to continuously manage lists. Expert System’s Cogito semantic technology leverages deep semantic analysis and the vast knowledge embedded in its semantic network to extract a virtually unlimited number of elements quickly and easily. Cogito features:
%
%Effective management of big data and growing volumes of unstructured information.
%Automatic identification and extraction of more than 20 standard entities.
%Customized selection and extraction of entities.
%Advanced exploration and analysis of concepts and relationships between documents.

\subsection{Scopo del progetto}
L'obiettivo del progetto sarà quello di creare un sistema esperto che sia in grado di estrarre le informazioni salienti da documenti giuridici inerenti istanze fallimentari.