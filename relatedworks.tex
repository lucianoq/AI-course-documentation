\section{Lavori correlati}
\label{relatedworks}

In \citet{mirizzilinked}, affrontando lo stesso problema, è stato utilizzato un VSM puro, trasformando il grafo RDF di partenza in un tensore 3-dimensionale di adiacenza.

Nel 2008, in \citet{passant2008combining}, gli autori introducevano la possibilità di utilizzare i Linked Open Data sia in un contesto di collaborative filtering (utilizzando il vocabolario \verb+foaf+) sia in recommender content-based (con le informazioni provenienti da \emph{dbpedia} e \emph{dbtune}).

In \citet{passant2009using}, gli autori effettuano un confronto tra un sistema di raccomandazione di tipo content-based attraverso l'utilizzo dei lod e un sistema di collaborative filtering.

In \citet{passant2010measuring}, invece, in un contesto differente dal nostro (artisti musicali), sono state introdotte le misure che verranno spiegate in seguito (\ref{measures}) e che verranno utilizzate come baseline per le nostre sperimentazioni. 

\nocite{freitas2012distributional}




\nocite{thalhammer2012leveraging}
