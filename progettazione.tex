\section{Progettazione}

\subsection{Obiettivo}
Come primo obiettivo, il sistema deve essere in grado di estrarre, da documenti testuali, alcune informazioni principali, quali:
\begin{itemize}
\item numero della pratica; %TODO
\item soggetto della richiesta di iscrizione;
\item quantità di denaro richiesta e tipologia.
\end{itemize}

Come feature aggiuntive, il sistema si può occupare dell'estrazione, su richiesta, di altre informazioni utili come:
\begin{itemize}
\item indirizzi email;
\item numeri di telefono;
\item nomi di comuni;
\item altre persone nominate nel testo;
\item etc.
\end{itemize}


\subsection{Input e Assunzioni}
Non avendo a disposizione dei documenti di input reali sui quali il sistema deve essere in grado di lavorare e, quindi, non conoscendone il formato o la composizione, abbiamo dovuto fare delle assunzioni e realizzare degli esempi di input ad hoc.
Per essere pronti a successivi adattamenti, abbiamo fatto le scelte meno vantaggiose, ipotizzando di non avere a disposizione alcuna strutturazione nel documento o metadati su di esso.
I nostri esempi di input sono delle semplicissime stringhe prive di \verb+newline+, tabulazioni, ordinamento nello spazio, tag di qualsiasi tipo.
Nel caso in cui in futuro dovessimo essere in possesso di alcune di queste informazioni, potremmo utilizzarle per migliorare le nostre regole.


\subsection{Strategia risolutiva}
Il problema che affronteremo rientra nell'ambito dell'\emph{Information Extraction}, un task di Intelligenza Artificiale che mira all'estrazione automatica di informazioni strutturate da documenti non-strutturati o semi-strutturati.

Nella maggior parte dei casi, questa attività riguarda l'elaborazione di testi scritti in linguaggio naturale (NLP).

Abbiamo deciso di servirci di alcune delle tecniche di analisi lessicale e sintattica già note e utilizzate in questo campo.

\subsubsection{Lexical Analysis}
Il testo in input (come già detto testo semplice in una stringa) subisce una pre-elaborazione lessicale, con l'obiettivo di ottenere una lista di token da passare al tagger (\ref{tagger}).

Questa pre-elaborazione principalmente si occupa di:
\begin{itemize}
\item eliminare i caratteri \emph{inutili} per i nostri scopi quali, ad esempio, \verb+-+, \verb+!+, \verb+?+;
\item normalizzare tutte le lettere in minuscolo;
\item produrre una lista di atomi che saranno i token in input al tagger.
\end{itemize}

\subsubsection{Syntactic Analysis}
\label{tagger}
Il tagger è la componente che lavora su liste di token alla ricerca di pattern noti per poter assegnare dei ruoli

\subsection{Oggetti del dominio}
Per poter estrarre le informazioni a più alto livello quali, ad esempio, \emph{soggetto} e \emph{quantità di denaro richiesta}, dobbiamo utilizzare numerosi oggetti, specifici del dominio giuridico e non.

\subsubsection{Persone}
Per l'individuazione delle persone fisiche presenti nel testo (oltre al soggetto della richiesta, anche il giudice, l'avvocato o altre persone nominate nel testo) è utile riuscire a distinguere un \fbox{nome} e un \fbox{cognome}, eventuali \fbox{titoli} quale 


\subsection{Individuazione delle funzioni}

\subsection{Individuazione delle relazioni}

\subsection{Dalla concettualizzazione alle regole}

