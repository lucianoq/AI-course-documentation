\section{Progettazione}

\subsection{Concettualizzazione}
Dai documenti testuali dovremo poter estrarre delle informazioni principali, quali:
\begin{itemize}
\item numero della pratica; %TODO
\item soggetto della richiesta di iscrizione;
\item quantità di denaro richiesta e tipologia.
\end{itemize}

Inoltre il sistema si deve occupare di estrarre, su richiesta, altre informazioni utili come:
\begin{itemize}
\item indirizzi email;
\item numeri di telefono;
\item nomi di comuni;
\item altre persone nominate nel testo;
\item etc.
\end{itemize}



\subsection{Oggetti del dominio}
Per poter estrarre le informazioni a più alto livello quali, ad esempio, \emph{soggetto} e \emph{quantità di denaro richiesta}, dobbiamo utilizzare numerosi oggetti, specifici del dominio giuridico e non.

\begin{lstlisting}
nome("Simone").
cognome("Rutigliano").
\end{lstlisting}


\subsection{Individuazione delle funzioni}

\subsection{Individuazione delle relazioni}

\subsection{Dalla concettualizzazione alle regole}

\subsection{Strategia risolutiva}