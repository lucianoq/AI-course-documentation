\section{Descrizione del sistema}

Il sistema presenta un'interfaccia grafica in grado di permettere l'interazione con il core del sistema scritto in Prolog, dando cosi la possibilità a qualunque tipo di stakeholders del sistema di utilizzarlo senza la necessità di dover interagire con il terminale, rendendo cosi le informazioni più leggibili e usabili; Oltre a questo, la creazione dell'interfaccia permette anche di evitare possibili errori dattilografici che si potrebbero avere in caso di interazione con il terminale.

Gli elementi principali dell’interfaccia, con i relativi pulsanti, la cui funzione e uso verranno descritti in dettaglio nel prossimo paragrafo, sono i seguenti (nella figura 1, i numeri delle aree corrispondono alle rispettive funzioni assegnate enumerate
nell’elenco seguente):
\begin{itemize}
  \item Sezione di inserimento del documento da cui si devono estrarre le informazioni
  \item Sezione di scelta delle informazioni da estrarre
  \item Sezione di visualizzazione dei risultati ottenuti
\end{itemize}
\subsection{Sezioni ed operazioni disponibili}
    \subsubsection{Inserimento}
    \label{Inserimento}
    In questa sezione viene data la possibilità di inserire il documento testuale da cui si vogliono estrarre le informazioni; con questa operazione non si fa altro che asserire un documento da dover poi essere processato dal core prolog del sistema.
    \subsubsection{Scelta dei tag}
    \label{ChoiceTag}
    In questa sezione si da la possibilità all'utente di filtrare i tag da voler estrarre dal documento attraverso la selezione/deselezione della checkbox corrispondente al tag; con questa operazione si vanno a selezionare quali saranno i tag che il core prolog deve etichettare nel documento.
    \subsubsection{Visualizzazione}
    \label{Visualization}
    In questa schermata invece verranno mostrate le informazioni che l'utente ha deciso di estrarre dal documento; inoltre le diverse tipologie di tag saranno evidenziate diversamente l'uno dall'altro attraverso l'ausilio di una colorazione dei tag. 

    \subsubsection{Reset delle condizioni iniziali}
    Tramite il pulsante \emph{Reset} si ripristinano le condizioni iniziali del sistema.In particolare, viene ripristinato lo stato iniziale:
    \begin{itemize}
      \item \emph{Interfaccia} : cancellando le textbox contenenti il documento inserito \ref{Inserimento} e i tag etichettati dal testo \ref{Visualization}, sia le scelte dei tag da effettuare \ref{ChoiceTag}.
      \item \emph{Core Prolog} : ritrattando il documento appena inserito nella sezione \ref{Inserimento}.
    \end{itemize}
    
\subsection{Interazione con l’utente} %TODO fare predicato per la scelta dei tag da fare
L’interazione avviene principalmente tramite l'interfaccia grafica descritta nella sezione precedente ma viene data anche la possibilità di interagire con il sistema anche senza l'ausilio di tale interfaccia, utilizzando direttamente l'interprete Prolog da terminale, previa consultazione del modulo principale del sistema denominato \emph{main.pl}; le funzionalità del sistema sono indipendenti dal metodo con cui si vuole interagire con il  sistema. Qualora si volesse tener traccia anche di come avviene la comunicazione tra java e prolog, è possibile visionare tale interazione all'interno della finestra di terminale da cui si è lanciato il sistema.

\subsubsection{JPL Library}
La libreria utilizzata per permettere la bidirezionalità della comunicazione tra Java e Prolog, è stata la \emph{JPL 3.1.4 alpha}\footnote{Scaricabile da \url{http://mvnrepository.com/artifact/jpl/jpl/3.1.4-alpha}} (\textbf{J}ava-calls-\textbf{P}rolog \textbf{L}ibrary)

Le API messe a disposizione dalla libreria JPL permettono la creazione di oggetti Java che andranno a inglobare gli elementi che saranno poi utilizzati lato Prolog. La struttura gerarchica delle classi presenti nella libreria è la seguente:
\begin{Verbatim}
Term
|
+--- Variable
|
+--- Compound
|      |
|      +--- Atom
|
+--- Integer
|
+--- Float

Query

JPLException
|
+-- PrologException
\end{Verbatim}

Le classi Term-based sono in grado di adattare al meglio la concreta sintassi strutturata dei termini di Prolog, in quanto non esiste una corrispondenza diretta con un particolare termine di Prolog; piuttosto, esistono diversi significati assunti dalla classe Term, si va dalla necessità di creare delle strutture da essere utilizzate all'interno di query Prolog, fino ad arrivare al tipo di rappresentazione, con relativa esplorazione, dei risultati ottenuti dalla query.
Per astrarre quindi il concetto di termine, la classe \emph{Term} è una classe astratta istanziata da una delle sue cinque possibili sottoclassi:
\paragraph{Compounds}
Un Compound è un Termine che contiene un nome e una sequenza di argomenti di tipo Term.

\begin{javacode}
Compound teacher_of = new Compound("teacher_of",
                  new Term[] {
                        new Atom("aristotle"),
                        new Atom("alexander")
                        }
                );
\end{javacode}

In questo esempio, la variabile java \emph{teacher\_of} si riferisce a una istanza di tipo \emph{Compound} che rappresenta il termine prolog.

\begin{prologcode}
  teacher_of(aristotle,alexander).
\end{prologcode}

\paragraph{Atom}
Un Atom è una specializzazione di \emph{Compound} avente zero parametri, il che vorrà dire che Atom sarà un Term contenente solo un nome.

\begin{javacode}
Atom aristotle = new Atom("aristotle");
Atom alexander = new Atom("alexander");
\end{javacode}

\paragraph{Variable}
Le Variabili sono dei Term aventi un nome identificativo il quale deve soddisfare la sintassi Prolog.

\begin{javacode}
Variable X = new Variable("X");  //  variabile X
Variable X = new Variable("_");  //  variabile "anonima"
Variable X = new Variable("_Y"); // variabile Y di cui non si vuole 
				 // conoscere il contenuto
\end{javacode}

\paragraph{Integer}
Un Integer è una specializzazione di \emph{Term} che mantiene la valorizzazione di tipo long di Java. Questa classe corrisponde al tipo integer di Prolog.

\begin{javacode}
jpl.Integer i = new jpl.Integer(5);
\end{javacode}

\paragraph{Floats}
Un Float è una specializzazione di \emph{Term} che mantiene la valorizzazione di tipo double di Java. Questa classe corrisponde al tipo float di Prolog sulla quale possono essere eseguite le operazioni aritmetiche.

\begin{javacode}
  jpl.Float f = new jpl.Float(3.14159265);
\end{javacode}

\paragraph{Queries}
Ogni istanza di Query contiene un \emph{Term}, il quale rappresenta il goal da dimostrare.

\begin{javacode}
	Term goal = new Compound("teacher_of",
	new Term[] {
		new Atom("aristotle"),
		new Atom("alexander")
	}
	);
  Query q = new Query( goal );
\end{javacode}
La query q in questo esempio rappresenta la query Prolog:

\begin{prologcode}
  ?- teacher_of(aristotle,alexander).
\end{prologcode}

\begin{Verbatim}
  JPL.setNativeLibraryDir(yapJPLPath);
  
  prolog.consult(new Atom("prolog/main.pl"));
  prolog.retractAll("domanda", 1);
  Term toAssert = new Compound("domanda", new Term[]{Util.textToTerm("\"" + textPane.getText() + "\"")});
  prolog.asserta(toAssert);
  java.util.Hashtable<String, Term>[] hashtables = prolog.allSolutions(new Compound("nextTag", new Term[]{new Variable("Tag")}));
  
\end{Verbatim}
The Prolog engine must be initialized before any queries are made, but this will happen automatically (upon the first call to a Prolog FLI routine) if it has not already been done explicitly.

\begin{javacode}
  public boolean consult(Atom atom) {
    Term t = new Compound("consult", new Term[]{atom});
    Query query = new Query(t);
    System.out.print("[Prolog] consult: " + t + " ");
    System.out.println(query.hasSolution() ? "succeeded" : "failed");
    return query.hasSolution();
  }
\end{javacode}

\begin{javacode}
  public void retract(Term term) {
    Term t = new Compound("retract", new Term[]{term});
    Query query = new Query(t);
    System.err.print("[Prolog] retract( " + term + " ) ");
    System.err.println(query.hasSolution() ? "succeeded" : "failed");
  }
\end{javacode}

\begin{javacode} 
  public void retractAll(String predicate, int arity) {  
    Term[] args = new Term[arity];
    for (int i = 0; i < args.length; i++)
    	args[i] = new Variable("_");
    Term[] termToRetract = new Term[]{ new Compound(predicate, args) };    
    Term t = new Compound("retractall", termToRetract);
    Query query = new Query(t);
    System.err.print("[Prolog] retract( " + predicate + " ) ");
    System.err.println(query.hasSolution() ? "succeeded" : "failed");
  }
\end{javacode}

\begin{javacode}
  public java.util.Hashtable oneSolution(Term t) {
    Query query = new Query(t);
    System.err.print("[Prolog] query: " + t + " ");
    System.err.println(query.hasSolution() ? "succeeded" : "failed");
    return query.oneSolution();
  }
\end{javacode}

\begin{javacode}
  public java.util.Hashtable[] nSolutions(Term t, long size) {
    Query query = new Query(t);
    System.err.print("[Prolog] query: " + t + " ");
    System.err.println(query.hasSolution() ? "succeeded" : "failed");
    return query.nSolutions(size);
  }
\end{javacode}

\begin{javacode}
  public java.util.Hashtable[] allSolutions(Term t) {
    Query query = new Query(t);
    System.err.print("[Prolog] query: " + t + " ");
    System.err.println(query.hasSolution() ? "succeeded" : "failed");
    return query.allSolutions();
  }
}
\end{javacode}
    

\subsection{Esempio di interazione}

\subsection{Utilizzo della funzione spiega domanda}

\subsection{Presentazione dei risultati}

\subsection{Utilizzo della funzione spiega ragionamento}
