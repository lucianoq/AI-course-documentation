\section{Interfaccia grafica ed esempi di esecuzione}

Il sistema presenta un'interfaccia grafica in grado di permettere l'interazione con il 


DepJaeger esibisce un’interfaccia grafica in grado di gestire le interazioni
con il nucleo del programma (scritto in Prolog), liberando quindi l’utente
dall’interazione col terminale, nonch`e di presentare le informazioni restituite
in maniera più leggibile e più comodamente fruibile. A parte l’evidente comodità d’uso, il sistema rende la fruizione anche intuitiva e scevra da errori dovuti ad errori di tipografia o di sequenza delle operazioni effettuate. Il gestore dell’interfaccia attuerà dei controlli tali da permettere l’attivazione di
varie funzioni solo quando le azioni richieste per farle funzionare sono state compiute, e guiderà l’utente verso la loro esecuzione. Gli elementi principali dell’interfaccia, con i relativi pulsanti, la cui funzione e uso verranno descritti in dettaglio nel prossimo paragrafo, sono i seguenti (nella figura 4.2, i numeri delle aree corrispondono alle rispettive funzioni assegnate enumerate
nell’elenco seguente):
1. Sezione di caricamento dei file
2. Sezione di analisi dei moduli
3. Sezione di analisi dei predicati
4. Sezione di output finale

\subsection{Schermata iniziale}

\subsection{Esempio di interazione}

\subsection{Utilizzo della funzione spiega domanda}

\subsection{Presentazione dei risultati}

\subsection{Utilizzo della funzione spiega ragionamento}